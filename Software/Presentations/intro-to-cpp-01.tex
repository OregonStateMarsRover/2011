\documentclass{beamer}
\usetheme{Warsaw}
\usecolortheme[RGB={205, 106, 0}]{structure}

\usepackage{beamerthemesplit}
\usepackage{multicol}
\usepackage{listings}


\title{An Introduction to Programming/C++ - Week 1}
\author{Ben Goska \\ Oregon State University}
\date{\today}

\begin{document}

\lstset{language=c++,
        numbers=left,
        showspaces=false,
	showstringspaces=false,
	basicstyle=\footnotesize
	}

\begin{frame}
	\titlepage
\end{frame}

\section[Outline]{}
\frame{\tableofcontents}

\section{The Tools}

\subsection{Compiler and Editor}

\begin{frame}
	\frametitle{Compiler and Editor}
	The main tool that is used for C++ development is the compiler. There
	are many avaliable, but we will use g++. To install g++:
	\begin{itemize}
		\item Windows: Download and install Dev-C++
		\item Linux: You should have g++, if not install it with the package manager
	\end{itemize}
	For editing code one of the following editors is recommended.
	\begin{itemize}
		\item gedit - windows + linux
		\item scite - windows + linux
		\item vi,vim,gvim - windows + linux
		\item notepad++ - windows
	\end{itemize}
\end{frame}

\subsection{Compiling}

\begin{frame}
	\frametitle{Compling}
	To compile a program using g++ you usually use the command line with a command 
	similar to: \\
	g++ file.cpp -o myProg \\
	which would compile file.cpp into a program called myProg
\end{frame}

\section{Programming Concepts}

\subsection{Pseudocode}

\begin{frame}[fragile]
	\frametitle{Writing down what we want to happen}
	Pseudocode: Writing down exactly what you want to happen \\
	Example of pseudocode to make peanut and jelly sandwich
	\begin{lstlisting}
	Get bread, peanut butter, jelly
	put peanut butter on bread
	put jelly on bread
	\end{lstlisting}
\end{frame}

\subsection{Variables}

\begin{frame}[fragile]
	\frametitle{Types of Variables}
	Variables can be used to store things, this includes strings (text), 
	whole numbers, decimal numbers, or even arrays (lists) of numbers. \\
	Example of pseudocode to do a little math $10 \frac {2}{14}+6$
	\begin{lstlisting}
	result = 0;
	result = result + 10;
	result = result * 2;
	result = result / 14;
	result = result + 6;
	\end{lstlisting}
\end{frame}

\subsection{Loops}
\begin{frame}[fragile]
	\frametitle{Loops and things}
	Loops are used to do something repeatedly, there are many types of loops.
	The most important kind of loop is a simple "while" loop. This type of
	loop runs while some condition is true, this is best shown in an example,
	here is an example of a loop to add up the numbers from 1 to 10.
	\begin{lstlisting}
	sum = 0;            //start the sum at 0
	i = 1;              //start counting at 1
	while (i < = 10) {   //loop until i > 10
	    sum = sum + i;
	    i = i + 1;
	}
	\end{lstlisting}
\end{frame}

\section{Introduction to C++}

\subsection{Basic C++}

\begin{frame}
	\frametitle{Pseudocode to C++}
	Moving from being able to write pseudocode to writing C++ is not too
	difficult, the main thing to remember is that C++ is very picky about
	how you write things.\\
	As a good rule you should start by writing pseudocode for what you want
	to accompilsh.
\end{frame}

\begin{frame}[fragile]
	\frametitle{Hello World}
	This is the most basic, but complete, C++ program.
	\begin{lstlisting}
	#include <iostream>

	int main(void) {
	    std::cout << "Hello World" << std::endl;
	    return 0;
	}
	\end{lstlisting}
\end{frame}

\begin{frame}[fragile]
	\frametitle{Hello World, explained}
	Line 1: This line includes the standard functions for outputting text to the screen. \\
	Line 3: This is the basic definition of a program, this is where code starts to be executed. \\
	Line 4: This puts the string "Hello World" into the output "cout", or writes "Hello World" to the console. \\
	Line 5: The program ends on this line, returning the value "0" to the operating system (this indicates no failure occured). 
\end{frame}

\subsection{Variables in C++}

\begin{frame}
	\frametitle{Types of variables}
	A few basic types in C++
	\begin{itemize}
		\item int - Integer, holds whole numbers (32-bit)
		\item char - character, holds one character (8-bit)
		\item double - Floating point, holds decimal numbers
		\item std::string - String, holds text
	\end{itemize}
	Modifiers that can be applied to variables
	\begin{itemize}
		\item unsigned - only holds non-negative numbers
		\item volatile - can change at any time (shouldn't ever be needed in normal code)
	\end{itemize}
\end{frame}

\begin{frame}
	\frametitle{Declaring variables}
	When you declare a variable you must include the following things:
	\begin{itemize}
		\item type
		\item name
	\end{itemize}
	And you may or may not have
	\begin{itemize}
		\item modifiers
		\item initial value
	\end{itemize}
\end{frame}

\begin{frame}[fragile]
	\frametitle{Declaring variables}
	Example: declare an unsigned integer, with an initial value of 0
	\begin{lstlisting}
	unsigned int myInteger = 0;
	\end{lstlisting}
	Note that variable names cannot have spaces, or special characters. 
	So you may use any letters (A-Z, a-z), numbers (0 - 9).
\end{frame}

\begin{frame}
	\frametitle{Using variables}
	All variables have operators that can be used, for numbers these 
	are the familiar operations, that include:
	\begin{itemize}
		\item $+$ Addition
		\item $-$ Subtraction
		\item $*$ Multiplication
		\item $/$ Division (NOTE: watch out for rounding)
		\item $\%$ Modulo (remainder of division)
		\item $=$ Assignment
	\end{itemize}
\end{frame}

\begin{frame}[fragile]
	\frametitle{Using variables}
	Now we can put all these together however we want to, including
	the use of parentheses, an example that expands on our our hello
	world program.
	\begin{lstlisting}
	#include <iostream>

	int main(void) {
	    int i = 0;  //make a variable to work with
	    i = (i + 10) * 2;
	    std::cout << i << std::endl;
	    i = i - 2;
	    std::cout << i << std::endl;
	    i = i / 2;
	    std::cout << i << std::endl;
	    return 0;
	}
	\end{lstlisting}
\end{frame}

\subsection{Loops in C++}

\begin{frame}
	\frametitle{Types of loops}
	C++ offers a number of different loop types, really they all perform
	the same function, though some are better for certain situations.
	\begin{itemize}
		\item while - The while loop is the simplest loop, it does something while a condition is true
		\item for - The for loop is just like a while loop but can initialize a variable and perform specific actions each loop
	\end{itemize}
\end{frame}

\begin{frame}[fragile]
	\frametitle{The while loop}
	This program prints out the numbers 1 to 10
	\begin{lstlisting}
	#include <iostream>

	int main(void) {
	    int i = 1;
	    while (i <= 10) {
	        std::cout << i << std::endl;
	        i++; //short hand for i = i + 1;
	    }
	    return 0;
	}
	\end{lstlisting}
\end{frame}

\begin{frame}[fragile]
	\frametitle{The for loop}
	This for loop does that same as the previous while loop.

	\begin{lstlisting}
	#include <iostream>

	int main(void) {
	    for(int i = 1; i <= 10; i++) {
	        std::cout << i << std::endl;
	    }
	    return 0;
	}
	\end{lstlisting}
\end{frame}

\subsection{If statements}

\begin{frame}[fragile]
	\frametitle{The if statement}
	The if statement is perhapse one of most useful statements in any programming language.
	
	\begin{lstlisting}
	#include <iosteam>

	int main(void) {
	    for(int i = 1; i <= 100; i++) {
	        if (i % 3 == 0) {
	            std::cout << i << std::endl;
	        }
	    }
	    return 0;
	}
	\end{lstlisting}
\end{frame}

\section{Exercises}

\begin{frame}
	\frametitle{Things to work on now}
	A few quick things to try out now
	\begin{itemize}
		\item Make a loop that adds up the numbers between 1 and 100, print the result
		\item Make a loop that adds up integers starting with 1 until the result is greater than 100
	\end{itemize}
\end{frame}

\end{document}
